\documentclass[12pt]{article}
\usepackage[utf8]{inputenc}
\usepackage[russian]{babel}
\usepackage{graphicx}
\usepackage{subcaption}
\graphicspath{ {./images/} }


\begin{document}

Дубровских Никита 221-361

\textit{\textbf{Вариант 7}}

\textit{\textbf{Задание 15.}}

\textit{По заданной матрице весов графа найти величину минимального
пути от вершины1x до каждой из вершин по алгоритму Дейкстры (в матричном
виде):}

\begin{center}
        \begin{tabular}{ |c|c|c|c|c|c|c| }
                \hline
				 & $x_1$ & $x_2$ & $x_3$ & $x_4$ & $x_5$ & $x_6$\\
                \hline
				x_1 & 0 & 4 & 9 & 8 & \infty & \infty \\
                \hline
				x_2 & \infty & 0 & 2 & \infty & \infty & \infty \\
                \hline
				x_3 & \infty & \infty & 0 & \infty & \infty & 3 \\
                \hline
				x_4 & 8 & 2 & 4 & 0 & 6 & \infty \\
                \hline
				x_5 & \infty & 2 & \infty & \infty & 0 & 3 \\
                \hline
				x_6 & \infty & \infty & \infty & 9 & \infty & 0 \\
                \hline
        \end{tabular}
\end{center}

\underline{Решение:}

Построим строку $T_1 = \{2,3,4,5,6\}$ - номера вершин до которых нужно вычислить длину пути и
$D^{(1)} = (0,\underline{4},9,8,\infty,\infty)$ - расстояния от $x_1$ до этих
вершин (первоначально совпадает с первой строкой матрицы весов).
Находим
минимальный элемент (подчеркнут) и удаляем его номер из строки Т.
Пересчитываем D по правилу: $D^{(s)} = (d_1^{(s)}, ...d_n^{(s)})$, где
$d_k^{(s+1)} = min\{d_k^{(s)}, d_j^{(s)} + w_j_k\}$, (т.е., если мы считаем k-ый
элемент в строк D, то
мы выбираем минимальное значение среди того элемента, который занимал эту
позицию в предыдущей строке D, а также среди всех сумм элементов столбца с
номером k матрицы весов и соответствующих, по порядку следования,
значений предыдущей строки D) если
$a_k \in T_{s+1}$, и $d_k^{(s+1)} = d_k^{(s)}$, если a_k \notin T_{s+1}$.

Получим:

$T_2 = \{3,4,5,6\}$.

$D^{(3)} = (0,4,\underline{6},8,14,12)$.

$T_3 = \{4,5,6\}$.

$D^{(4)} = (0,4,6,\underline{8},14,9)$.

$T_4 = \{5,6\}$.

$D^{(5)} = (0,4,6,8,14,9)$.

Строка $D^{(5)}$ не отличается от $D^{(4)}$, поэтому решение закончено даже несмотря на то, что в строке T остались элементы.

Ответ: минимальные расстояния от вершины 1 до всех остальных: (0,4,6,8,14,9).

\end{document}
