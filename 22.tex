\documentclass[12pt]{article}
\usepackage[utf8]{inputenc}
\usepackage[russian]{babel}
\usepackage{graphicx}
\usepackage{amsmath}
\usepackage{nicematrix,tikz}
\graphicspath{ {./images/} }


\begin{document}

Дубровских Никита 221-361

\textit{\textbf{Вариант 7}}

\textit{\textbf{Задание 22.}}

\textit{При помощи венгерского алгоритма решите задачу на максимум
для двудольного графа, заданного матрицей:}

\begin{center}
$C = \begin{pmatrix}
	4 & 3 & 5 & 6 \\
	3 & 4 & 8 & 7 \\
	5 & 7 & 5 & 6 \\
	4 & 6 & 8 & 4
\end{pmatrix}$.
\end{center}

\underline{Решение:}

Поскольку решается задача на максимум, то нам потребуется сделать
один дополнительный шаг: найдем максимальный элемент – 8 и отнимем
каждый элемент от него. Получим:

\begin{center}
$C = \begin{pmatrix}
	4 & 5 & 3 & 2 \\
	5 & 4 & 0 & 1 \\
	3 & 1 & 3 & 2 \\
	4 & 2 & 0 & 4
\end{pmatrix}$.
\end{center}

Задача свелась к решению задачи на минимум.

Теперь проведем редукцию по строкам (найдем в каждой строке
наименьший элемент и вычтем его из всех элементов данной строки):

\begin{center}
$C = \begin{pmatrix}
	2 & 3 & 1 & 0 \\
	5 & 4 & 0 & 1 \\
	2 & 0 & 2 & 1 \\
	4 & 2 & 0 & 4
\end{pmatrix}$.
\end{center}

Теперь проведем редукцию по столбцам:

\begin{center}
$C = \begin{pmatrix}
	0 & 3 & 1 & 0 \\
	3 & 4 & 0 & 1 \\
	0 & 0 & 2 & 1 \\
	2 & 2 & 0 & 4
\end{pmatrix}$.
\end{center}

Поскольку нули выбрать не получается нам нужно вычеркнуть все нули
минимальным количеством горизонтальных и/или
горизонтальных линий:

\[
	C = 
\begin{pNiceMatrix}
	0 & 3 & 1 & 0 \\
	3 & 4 & 0 & 1 \\
	0 & 0 & 2 & 1 \\
	2 & 2 & 0 & 4
% \CodeAfter \tikz [black] \draw (1.5-|1) -- (1.5-|last) (1-|3.5) -- (last-|3.5) ;
	\CodeAfter \tikz [black] \draw (3.5-|1) -- (3.5-|last)
	(1-|3.5) -- (last-|3.5)
	(1.5-|1) -- (1.5-|last);
\end{pNiceMatrix}
\]

Среди всех невычеркнутых элементов находим минимальный. Отнимаем
его от всех невычеркнутых элементов и прибавляем в местах пересечения
линий, те элементы, через которые проходит только одна линия не трогаем.

Получим:

\begin{center}
$C = \begin{pmatrix}
	0 & 3 & 2 & 0 \\
	2 & 3 & 0 & 0 \\
	0 & 0 & 3 & 1 \\
	1 & 1 & 0 & 3
\end{pmatrix}$.
\end{center}

Теперь смотрим, можно ли здесь выбрать нули (в каждой строке или
каждом столбце ровно один ноль). Такое возможно:

\begin{center}
$C = \begin{pmatrix}
	\tikz \node[draw,circle, inner sep=0pt, minimum size=4mm, color=red]{0} & 3 & 2 & 0 \\
	2 & 3 & 0 & \tikz \node[draw,circle, inner sep=0pt, minimum size=4mm, color=red]{0} \\
	0 & \tikz \node[draw,circle, inner sep=0pt, minimum size=4mm, color=red]{0} & 3 & 1 \\
	1 & 1 & \tikz \node[draw,circle, inner sep=0pt, minimum size=4mm, color=red]{0} & 3
\end{pmatrix}$.
\end{center}

Тогда ответ:

\begin{center}
$C = \begin{pmatrix}
	\tikz \node[draw,circle, inner sep=0pt, minimum size=4mm, color=red]{4} & 3 & 5 & 6 \\
	3 & 4 & 8 & \tikz \node[draw,circle, inner sep=0pt, minimum size=4mm, color=red]{7} \\
	5 & \tikz \node[draw,circle, inner sep=0pt, minimum size=4mm, color=red]{7} & 5 & 6 \\
	4 & 6 & \tikz \node[draw,circle, inner sep=0pt, minimum size=4mm, color=red]{8} & 4
\end{pmatrix}$.
\end{center}

\end{document}
